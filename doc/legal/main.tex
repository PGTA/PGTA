% rubber : module pdftex
\documentclass{tufte-handout}
\usepackage[pdftex]{graphicx}
% some common packages
\usepackage{xspace}
\usepackage{url} \urlstyle{same}

\newcommand{\etc}{\emph{etc.}\xspace}
\newcommand{\ie}{\emph{i.e.}\xspace}
\newcommand{\eg}{\emph{e.g.}\xspace}

\setcounter{secnumdepth}{3}

% default values for macros
\newcommand{\teamname}{Procedurally Generated Transitional Audio}
\newcommand{\projectdesc}{}
\newcommand{\studentnames}{Keefer Davies, Andrew Dawidek, Jordan Cristiano, Stephen Shi}

\newcommand{\contactemail}{pgta@gmail.com}

% entrepreneur, research, society fun
\newcommand{\category}{entrepreneur}

\newcommand{\teamurl}{url}
\newcommand{\equipment}{}

\renewcommand{\teamname}{Procedurally Generated Transitional Audio (PGTA)}

\renewcommand{\projectdesc}{
This project provides a solution that removes the need for using pre-recorded compositions as 
music/ambience in games. This is achieved by dynamically modifying a multitude of sound-state parameters that 
are used as input to a scheduling algorithm responsible for producing structured ambient music from pieced together 
audio clips. Since the audio generation is dynamic and parameter driven, the music it produces is able 
to seamlessly respond and adapt to in-game events with minimal latency.
}

\renewcommand{\studentnames}{%
Keefer Davies, Jordan Cristiano, Andrew Dawidek, Stephen Shi}

\renewcommand{\contactemail}{group@pgta.ca}

% entrepreneur, research, society, fun
\renewcommand{\category}{Entertainment}

\renewcommand{\teamurl}{http://www.pgta.ca}

% special equipment requests, e.g.:
% - more than one network drop
% - flying helicopter
% - kinect: requires space for people to use it for demo
% - etc.
\renewcommand{\equipment}{N/A}



\title{Legal/Ethical/Environmental/Social Report}
\author{\teamname\marginnote{\studentnames}}
\date{Winter 2015}

\begin{document}
\maketitle

%%%%%%%%%%%%%%%%%%%%%%%%%%%%%%%%%%%%%%%%%%%%%%%%%%%%%%%%%%%%%%%%%%%%%%%%

\marginnote{%
Comment on the legal, social, ethical, and environmental aspects of
your projects, as applicable. For most projects legal is the main
issue, and in most cases software licences (\ie, copyright) is the
primary concern.

Your goal here is to demonstrate awareness of the issues. You are not
a lawyer, and you do not need to provide a legal judgement: if there
are grey areas then you can just say that a legal opinion should be
sought on the matter.
}% end marginnote


Video games are a unique form of entertainment that require a high degree of user interaction. The nondeterministic 
nature of this interaction makes it difficult to synchronize gameplay with a soundtrack. Ambient audio is used in video games as a
means of enhancing immersion, and having this ambient audio react to gameplay furthers this immersion. One way to achieve
this level of integration between ambient audio and gameplay is to procedurally generate sound based on in-game events.
Though this technique has been explored before
by game developers, all of the work has been proprietary and thus not accessible to the public. \textbf{PGTA} provides 
an open-source solution to procedurally generate audio targeted towards integration with game engines. In addition to 
procedural generation of tracks using pre-recorded audio samples, PGTA also offers a means of dynamically 
transitioning between tracks. The PGTA engine also comes with a custom editor for PGTA projects providing an intuitive method
to create tracks and transitions while requiring minimal technical skills. PGTA provides an easily accessible and user friendly
approach to creating procedurally generated ambient audio. 



%-----------------------------------------------------------------------
% table of contents with reduced spacing
\newlength{\dfrtmpparskip}
\providecommand{\tighttableofcontents}{
    \setlength{\dfrtmpparskip}{\parskip}
    \setlength{\parskip}{0pt plus 1ex}
    \tableofcontents
    \setlength{\parskip}{\dfrtmpparskip}
}
\tighttableofcontents
%-----------------------------------------------------------------------

\clearpage
%%%%%%%%%%%%%%%%%%%%%%%%%%%%%%%%%%%%%%%%%%%%%%%%%%%%%%%%%%%%%%%%%%%%%%%%
\section{Intellectual Property} 

`Intellectual property' is a recent term used to refer to four
historically distinct sets of laws: patent, copyright, trademark, and
trade secrets.\cite{Stallman-not-ipr} 

University of Waterloo Policy 73\cite{UW-Policy73} is unique in that
it allows you to retain ownership of all of the intellectual property
that you create at school. Very few (if any) other schools have such a
policy: usually the university claims ownership of all intellectual
material created as part of university business.\marginnote{According
to Policy~73 UW does retain ownership of final exams by classifying
them as a faculty administrative task rather than as product of
teaching.} 
%
This policy gives you tremendous freedom and makes writing this report
much simpler. 

%%%%%%%%%%%%%%%%%%%%%%%%%%%%%%%%%%%%%%%%%%%%%%%%%%%%%%%%%%%%%%%%%%%%%%%%
\subsection{Copyright} 

Copyright is the branch of law most commonly associated with software,
as software is written work. Copyright is the legal basis of all
open-source software licences. You should be able to answer the
following questions:

\begin{itemize}

    \item What are the licenses attached to the software you are using?
    \begin{itemize}
        \item SDL2 - zlib License
        \item protobuf - New BSD License
    \end{itemize}

%-----------------------------------------------------------------------
\marginnote{\textsc{gpl} compatibility is discussed on the
\textsc{gnu} web site. There are many other online resources on this
topic.

\parbox{3in}{%
\url{http://www.gnu.org/licenses/quick-guide-gplv3.html}

\url{http://www.gnu.org/licenses/license-list.html}}}
%-----------------------------------------------------------------------

    \item Are %
    \marginnote{Please provide appropriate references for your claims,
    \eg: 

    \url{http://www.softwarefreedom.org/resources/2007/gpl-non-gpl-collaboration.html}

    \url{http://www.apache.org/licenses/GPL-compatibility.html}}
    %
    the pieces of software that you are using license
    compatible with each other? 

    Yes, all libraries used can be redistributed in source or binary form with or without modification in a commercial
    or non-commercial environment. 

    \item What license options are available for your project? If you
    are linking with \textsc{gpl} software then your project must be
    \textsc{gpl}, \etc

    There are a number of open-source licenses available for our project such as, MIT, zlib, GPL, New BSD. 

    \item What license are you choosing for your project? Why?

    The MIT open-source licence is being used for PGTA. This license fits the criteria of being open-source 
    software with minimal restrictions on usage for third-parties. The MIT license also protects against
    accidental damages that result from the use of the PGTA engine. We as developers of PGTA do not want to be 
    held responsible for any such damages. 

    \item Does your project involve, or appear to involve, 
    %
    \marginnote{There is a rich legal history on this topic that you
    could briefly reference, \eg, Napster, Morpheus, MegaUpload, \etc}
    %
    sharing or capture of third party data?  Third party data should
    be understood broadly, including at least recorded music or
    movies, Google maps data, Yelp local business data, \etc.  
    What are the terms of service/usage for the data?

    Not applicable

    \item Who will retain ownership of the copyrights on your software
    after you graduate? You? Your customer? Someone else?

    The developers of the PGTA engine will retain ownership of all aspects of the software upon graduation.

\end{itemize}


%%%%%%%%%%%%%%%%%%%%%%%%%%%%%%%%%%%%%%%%%%%%%%%%%%%%%%%%%%%%%%%%%%%%%%%%
\subsection{Patent} 

\begin{itemize}

    \item Is there patentable material in your project? Have you
    applied? Are you applying?

    The scheduling algorithm with respect to how it handles transitions between tracks is patentable. As a group there
    currently is no intention to obtain a patent for any aspect of the PGTA engine. 

    \item Is the 
    %
    \marginnote{For example, many \textsc{codecs} are patent
    encumbered in most of the developed world, and hence do not ship
    with many \textsc{gnu}/Linux distributions.}
    %
    software that you are using patent encumbered in certain
    countries? Does this restrict the ability to redistribute your
    software? 

    Our research was inconclusive, thus consultation with a legal expert is required for this matter. 

\end{itemize}


%%%%%%%%%%%%%%%%%%%%%%%%%%%%%%%%%%%%%%%%%%%%%%%%%%%%%%%%%%%%%%%%%%%%%%%%
\subsection{Trademark}

\begin{itemize}
    \item Procedurally Generated Transitional Audio name 
    \item PGTA acronym
    \item PGTA logo
\end{itemize}

%%%%%%%%%%%%%%%%%%%%%%%%%%%%%%%%%%%%%%%%%%%%%%%%%%%%%%%%%%%%%%%%%%%%%%%%
\subsection{Trade Secrets}

Not applicable

%%%%%%%%%%%%%%%%%%%%%%%%%%%%%%%%%%%%%%%%%%%%%%%%%%%%%%%%%%%%%%%%%%%%%%%%
\subsection{Export Controls} 

In some countries, such as the United States, some technologies, such
as cryptography, are restricted by export controls. For example, this
is why OpenSSH and OpenBSD are developed in
Canada.\marginnote{\url{http://www.openbsd.org/crypto.html}}.
%
If technology used in your project is subject to export or import
controls in Canada, the United States, or the United Kingdom, please
discuss.


Not applicable 



%%%%%%%%%%%%%%%%%%%%%%%%%%%%%%%%%%%%%%%%%%%%%%%%%%%%%%%%%%%%%%%%%%%%%%%%
\subsection{End User License Agreement or Terms of Service}

If your project requires an End User License Agreement or a Terms of
Service agreement, please provide and discuss it here.


The End User License Agreement is outlined in the MIT license. The PGTA engine is to be provided as is without 
warranty of any kind. The developers and copyright holders of PGTA are not responsible for any damages or liability
arising form the use of the PGTA engine. 

%%%%%%%%%%%%%%%%%%%%%%%%%%%%%%%%%%%%%%%%%%%%%%%%%%%%%%%%%%%%%%%%%%%%%%%%
\newpage
\section{Privacy} 

\begin{itemize}

\item Jurisdiction: Where will your software be run? Where will its
users be? Which jurisdictions should be considered?

This tool is distributed to the open-source community and thus is not bound to a specific jurisdiction. 


\item Canada: \emph{Personal
Information Protection and Electronic Documents Act (PIPEDA)}.
\marginnote{Some provinces, such as Ontario, Quebec, Alberta, and
British Columbia, have their own privacy legislation. This provincial
legislation largely mirrors the federal legislation but is not
identical, and in some cases the differences might count.}

PGTA does not collect personal information and therefore cannot expose any personal information. 


\item USA: \emph{Health Insurance Portability and Accountability Act
(HIPAA)} 

Not applicable

\item Europe: \emph{Data Protection Directive}

Not applicable

\end{itemize}


%%%%%%%%%%%%%%%%%%%%%%%%%%%%%%%%%%%%%%%%%%%%%%%%%%%%%%%%%%%%%%%%%%%%%%%%
\newpage
\section{Industry Standards, Regulations, Norms}
\begin{itemize}
    \item PGTA provides a C application programming interface (compatible with all languages)
    \item PGTA is designed to work with common game engine architectures (game loop with update calls)
    \item PGTA makes us of streaming Pulse-code Modulation (PCM) 
    \item PGTA uses the SI unit for time (seconds)
\end{itemize}

%%%%%%%%%%%%%%%%%%%%%%%%%%%%%%%%%%%%%%%%%%%%%%%%%%%%%%%%%%%%%%%%%%%%%%%%
\newpage
\section{Ethics}

%%%%%%%%%%%%%%%%%%%%%%%%%%%%%%%%%%%%%%%%%%%%%%%%%%%%%%%%%%%%%%%%%%%%%%%%
\subsection{Professional Ethics (PEO)}

\marginnote{\url{http://www1.peo.on.ca/Ethics/code_of_ethics.html}}

\begin{itemize}

\item Duty to clients.

PGTA does not and has no intentions of collecting any personal information from its users. All representatives of PGTA
will exercise professional integrity when dealing with clients and will not engage in discrimination or favouritism. 

\item Duty to self.

The members of PGTA owe it to themselves to have satisfaction with the final product. The product should be complete 
and usable in practice. At all times members of PGTA must remain honourable, respectful, fair and honest to all related
parties. 

\end{itemize}

%%%%%%%%%%%%%%%%%%%%%%%%%%%%%%%%%%%%%%%%%%%%%%%%%%%%%%%%%%%%%%%%%%%%%%%%
\subsection{Philosophical Ethics}

\marginnote{There are three main approaches to ethics in philosophy.
%
The \emph{virtue} approach says one should help a person in need
because it exercises the virtues of charity and benevolence. 
%
The \emph{rules} approach says one should help a person in need
because it follows the golden rule: do unto others as you would have
them do unto you. 
%
The \emph{consequences} approach says one should help a person in need
because it makes the world a better place.
%
In many cases all three approaches agree. Things get interesting when
the different approaches do not agree.
}

\begin{itemize}

\item Virtue

The core developers of this software will provide ongoing support to clients. This support will cover bug fixing, 
compatibility and integration issues, addressing feature requests and offering instruction to those in need. When 
communicating with clients, representatives of PGTA will conduct themselves professionally.

\end{itemize}

%%%%%%%%%%%%%%%%%%%%%%%%%%%%%%%%%%%%%%%%%%%%%%%%%%%%%%%%%%%%%%%%%%%%%%%%
\newpage
\section{Social}

Not applicable

%%%%%%%%%%%%%%%%%%%%%%%%%%%%%%%%%%%%%%%%%%%%%%%%%%%%%%%%%%%%%%%%%%%%%%%%
\section{Environmental} 

Not applicable


%%%%%%%%%%%%%%%%%%%%%%%%%%%%%%%%%%%%%%%%%%%%%%%%%%%%%%%%%%%%%%%%%%%%%%%%
\bibliographystyle{plainnat}
\nobibliography{main}

\end{document}

